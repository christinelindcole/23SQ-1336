% Instructions to change to html version:
% Comment out:
%  minipage, multicols,columnbreak, mathbf, hrule
% Replace \$$ with \[ and $ with \(
% Enclose graphics in figure environments and add captions
% Re-tag \df environments as sections, subsections, etc.
% Command Line Code to Create html version:
%First: pdflatex -shell-escape filename.tex                                   
%Second, for each figure: inkscape "filename-figure1.pdf" -o "filename-figure1.png"
% Third: htlatex filename.tex "ht5mjlatex.cfg, charset=utf-8" " -cunihtf -utf8"

\documentclass[10pt]{article}

%\usepackage{tikz, pgf,pgfplots,wasysym,array}
%\usepackage{wasysym,array}

\usepackage{amsmath,amssymb}

\ifdefined\HCode
  \def\pgfsysdriver{pgfsys-tex4ht-updated.def}
\fi 
%\ifdefined\HCode
%  \def\pgfsysdriver{pgfsys-dvisvgm4ht.def}
%\fi 
\usepackage{tikz}
\usetikzlibrary{calc,decorations.markings,arrows}
\usepackage{pgfplots}

\pgfplotsset{compat=1.12}
\usepackage{myexternalize}
\usetikzlibrary{calc,decorations.markings,arrows}
\usepackage{framed}
\usepackage[none]{hyphenat}

\input{../../../common/1336_header_test.tex}
% % !tex root = ./exam_2.tex
\usepackage[nomessages]{fp}% 
\usepackage{rotating}
\usepackage{graphicx}
\usepackage{sectsty}
\usepackage{xparse}
\usepackage{tikz}
\usepackage{pgf,pgfplots}
	% For the histogram ?
	\pgfdeclarelayer{background}% determine background layer
	\pgfsetlayers{main,background}% order of layers

\usepackage{tkz-berge}
\usetikzlibrary{calc,graphs,arrows,backgrounds,decorations.pathreplacing}
%Not found:\usetikzlibrary{graphs.standard,quotes}
\usetikzlibrary{shapes}
		\tikzset{
		  dot hidden/.style={},
		  line hidden/.style={},
		  dot colour/.style={dot hidden/.append style={color=#1}},
		  dot colour/.default=black,
		  line colour/.style={line hidden/.append style={color=#1}},
		  line colour/.default=black
		}
\NewDocumentCommand{\rot}{O{45} O{1em} m}{\makebox[#2][l]{\rotatebox{#1}{#3}}}%


\begin{document}

\graphicspath{{Series_Strategies/}}

\newcommand{\an}{\lbrace a_n \rbrace}
\newcommand{\Sum}{\sum_{n=1}^\infty }
\newcommand{\SumZero}{\sum_{n=0}^\infty }

\everymath{\displaystyle}

\renewcommand{\myTitle}{	MATH 1336: Calculus III}

\renewcommand{\mySubTitle}{Section 8.4, Part 2: Ratio Test \& Root Test}%, and\\ Strategies for Testing Series}% \vspace*{-.25in}}
%~\hfill Name: \underline{~~~~~~~~~~~~~~~~~~~~~~~~~~~~~~~~~~~~~~~~~~~~~~~}

%\lectTitle{\vspace*{-.5in}\myTitle}{\vspace*{.1in}\mySubTitle \vspace*{-.2in}}

\title{\mySubTitle}\date{}
\maketitle



%\hspace*{-.8in}%\begin{minipage}{1.25\textwidth}

\setlength{\columnseprule}{.4pt}
\setlength{\columnsep}{3em}

%\begin{framed}
\section*{Section 8.4 - More Series Tests!: }
Both of the tests listed below can be used on series that have some negative terms.\\
Both tests are built on the idea of comparing how fast the terms are going to zero with geometric series.\\
\textbf{Note: Both tests will be inconclusive when applied to p-series or series with terms that are rational functions!}\\


%\hrule
\vspace*{.1in}

%\begin{multicols}{2}



\subsection*{Ratio Test:}
Consider the series \(\sum a_n\). 
 \begin{enumerate}
 \item If \(\lim_{n\rightarrow \infty} \left| \frac{a_{n+1}}{a_n}\right| = L < 1\):\\
  then \(\sum a_n\) is absolutely convergent.\\ 
  \item If \(\lim_{n\rightarrow \infty} \left| \frac{a_{n+1}}{a_n}\right| = L > 1\)
  -OR- 
  \(\lim_{n\rightarrow \infty} \left| \frac{a_{n+1}}{a_n}\right| =\infty\):\\
   then \(\sum a_n\) is divergent.\\
\item \(\lim_{n\rightarrow \infty} \left| \frac{a_{n+1}}{a_n}\right| = 1\):\\
 then the test is is inconclusive.\\
  (Try something else!)
 \end{enumerate} 
% ~\\
 \underline{\textbf{Useful for Series:}}\\
 that contain a factorial
 %%\endmulticols}%\( \\~\\~\\
 
% \vspace*{.25in}
 %\columnbreak
\subsection*{Root Test:}
Consider the series \(\sum a_n\). 
 \begin{enumerate}
 \item If \(\lim_{n\rightarrow \infty} \sqrt[n]{|a_n|}= L < 1\):\\
  then \(\sum a_n\) is absolutely convergent. %\vspace*{.1in}
  \item If \(\lim_{n\rightarrow \infty} \sqrt[n]{|a_n|} = L > 1\)
  -OR- 
  \(\lim_{n\rightarrow \infty} \sqrt[n]{|a_n|}=\infty\):\\
   then \(\sum a_n\) is divergent.  %\vspace*{.15in}
\item \(\lim_{n\rightarrow \infty} \sqrt[n]{|a_n|} = 1\):\\
 then the test is is inconclusive.\\
  (Try something else!)
 \end{enumerate} 
  \underline{\textbf{Useful for Series:}}\\
 where the whole expression is raised to the  \(n^{th}\) power
 % \vspace*{.25in}

%\endmulticols}

%\end{framed}

%\end{minipage}

\section*{Examples we will work through together:}
%\setlength{\columnseprule}{.4pt}
%\setlength{\columnsep}{3em}
%%\begin{multicols}{2}

Determine the Convergence of the following Series:
\begin{enumerate}%[{Example }1:]
%\addtocounter{enumi}{3}

\item  \(\qquad\Sum \frac{10^n}{n!}\)
\vfill

\item  \(\qquad\Sum \frac{(-2)^n}{n^2}\)
\vfill

\item  \(\qquad\Sum \frac{10^n}{n!}\)
\vfill


\pagebreak

\hspace{-.75in} Use the Ratio Test on the following Series:

\item  \(\qquad\Sum \frac{1}{n^2}\)
\vfill

\item  \(\qquad\Sum \frac{1}{n}\)

%\vspace*{1in} %
\vfill

%\pagebreak
\item \(\qquad\Sum	\frac{n^n}{n!}\)
%\vspace*{.75in} %
\vfill

%%\columnbreak
%\vspace*{.1in}
\hspace*{-.75in}Use the Root Test on the following Series:

\item  \(\qquad\Sum \frac{(-2)^n}{n^n}\)

%\vspace*{1in} %
\vfill
\item \(\qquad\Sum	\left(\frac{n^2+1}{2n^2+1}\right)^n\)
%\vspace*{.75in} %
\vfill

\end{enumerate}

%%\endmulticols}

\pagebreak

\section*{Problems for Group Work:}

For each of the following series, apply either the Ratio Test or the Root Test, or state that one of those tests is inconclusive.


\begin{enumerate}

\item \(\qquad\SumZero (-1)^{n}\frac{n!}{1000^n}\) \label{prob1}
% Problem \ref{prob1}: Diverges

\vfill

\item \(\qquad\Sum \frac{2^n}{n^{2n}} \) \label{prob2}
% Problem \ref{prob2}: Converges Absolutely

\vfill

\item \(\qquad\Sum \frac{n}{n^2+1}\) \label{prob3}
% Problem \ref{prob3}: Ratio Test is Inconclusive

\vfill

%\item \(\qquad\Sum \frac{n^{40}}{n!}\) \label{prob4}
%% Problem \ref{prob4}: Converges Absolutely
%
%
%\vfill

\item \(\qquad\SumZero \left(\frac{5n}{3n+1}\right)^n\) \label{prob5}
% Problem \ref{prob5}: Diverges


\vfill


%\vfill

%\rotatebox{180}{
%\begin{minipage}{\textwidth}
\subsection*{Answers:}
\textbf{Problem \ref{prob1}:} Diverges, 
\textbf{Problem \ref{prob2}:} Converges Absolutely, 
\textbf{Problem \ref{prob3}:} Ratio Test is Inconclusive, 
%\textbf{Problem \ref{prob4}:} Converges Absolutely,
\\\textbf{Problem \ref{prob5}:} Diverges

%\end{minipage}
%}



\end{enumerate}


%
%\pagebreak
%
%\input{Series_Strategies/convergence_test_summary}
%
%\pagebreak
%
%\section*{Problems for Group Work}
%
%For each of the following series, state which test you would use to determine the convergence or divergence behavior, and explain why.
%
%(You do not have to carry out the test in detail, but follow the argument long enough to make sure your reasoning would work.)
%
%\begin{enumerate}
%%\begin{multicols}{2}
%
%\item \(\Sum \frac{n-1}{2n+1}\)\vspace*{.25in}
%
%\item \(\Sum \frac{\sqrt{n^3+1}}{3n^3+4n^2+2}\)\vspace*{.25in}
%
%\item \(\Sum ne^{-n^2}\)\vspace*{.25in}
%
%\item \(\Sum (-1)^{n}\frac{n^3}{n^4+1}\)\vspace*{.25in}
%
%\item \(\Sum \frac{2^k}{k!}\)\vspace*{.25in}
%
%\item \(\Sum \frac{1}{2+3^n}\)\vspace*{.25in}
%%\endmulticols}
%
%\section*{Additional Series For Practice}
%Solutions will be posted on Canvas. (Don't bother clicking in the blue boxes below, they won't work.)
%
%\includegraphics[width=\textwidth]{Stewart_Probs}
%
%\end{enumerate}

\end{document}
