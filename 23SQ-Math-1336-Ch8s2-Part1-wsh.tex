% Instructions to change to html version:
% Comment out:
%  minipage, multicol, columbreak, mathbf, hrule, framed
% Replace $$ with \[ and $ with \(
% Enclose graphics in figure environments and add captions
% Re-tag \df environments as sections, subsections, etc.
% Command Line Code to Create html version:
%First: pdflatex -shell-escape filename.tex                                   
%Second, for each figure: inkscape "filename-figure1.pdf" -o "filename-figure1.png"
% Third: htlatex filename.tex "ht5mjlatex.cfg, charset=utf-8" " -cunihtf -utf8"
\documentclass[10pt]{article}

%\usepackage{tikz, pgf,pgfplots,wasysym,array}
%\usepackage{wasysym,array}

\usepackage{amsmath,amssymb}

\ifdefined\HCode
  \def\pgfsysdriver{pgfsys-tex4ht-updated.def}
\fi 
%\ifdefined\HCode
%  \def\pgfsysdriver{pgfsys-dvisvgm4ht.def}
%\fi 
\usepackage{tikz}
\usetikzlibrary{calc,decorations.markings,arrows}
\usepackage{pgfplots}

\pgfplotsset{compat=1.3}
\usepackage{myexternalize}
\usetikzlibrary{calc,decorations.markings,arrows}
\usepackage{framed}
\usepackage[none]{hyphenat}

\input{../../../common/1336_header_test.tex}
\usepackage[none]{hyphenat}
\begin{document}

\newcommand{\an}{\lbrace a_n \rbrace}
\newcommand{\sn}{\lbrace s_n\rbrace}
\newcommand{\Sum}{\sum_{n=1}^\infty }

\everymath{\displaystyle}

\renewcommand{\myTitle}{\vspace*{-.25in}	MATH 1336: Calculus III}

\renewcommand{\mySubTitle}{Section 8.2, Part 1: Intro to Infinite Series}
%~\hfill Name: \underline{~~~~~~~~~~~~~~~~~~~~~~~~~~~~~~~~~~~~~~~~~~~~~~~}

%\lectTitle{\vspace*{-.25in}\myTitle}{\vspace*{.1in}\mySubTitle \vspace*{-.2in}}


\title{\mySubTitle}\date{}
\maketitle

\hspace*{-.8in}%\begin{minipage}{1.25\textwidth}

\setlength{\columnseprule}{.4pt}
\setlength{\columnsep}{3em}

%\begin{framed}
%\textbf{New Tests \& Theorems:} 

%\begin{multicols}{2}

\section*{Infinite Series Definitions: }~\\
An \textbf{infinite series} is the sum of all of the infinitely many terms in an infinite sequence:

\[
\Sum a_n = a_1 + a_2 + a_3 + \ldots + a_n + \ldots
\]

The \(n^{th}\)\textbf{partial sum}, \(s_n\), is the sum of the first \(n\) terms:


\begin{align*}
s_1 &= a_1\\
s_2 &= a_1 + a_2\\
s_3 & = a_1 + a_2 + a_3\\
\vdots &\\
s_n & = a_1 + a_2 + a_3+ \ldots + a_{n-1} + a_n = \sum_{i=1}^{n} a_i
\end{align*}

%\columnbreak

The partial sums themselves form a sequence: the sequence of partial sums \(\sn\).\\


If \(\sn\) converges, and if \(\lim_{n\rightarrow\infty} s_n = S\) exists and is finite, then we say that the series is \textbf{convergent}, and the number \(S\) is called the \textbf{sum} of the series:\\

\[
\Sum a_n = a_1 + a_2 + a_3 + \ldots + a_n + \ldots = \lim_{n\rightarrow\infty} s_n =S
\]

If \(\sn\) diverges, then we say that the series is \textbf{divergent}.\\

~\\

\textbf{Key Idea:}\\
If we can find a formula for the \(n^{th}\) partial sum, \(s_n\), somehow, then we can check to see if \(\sn\) converges using the tools that we have already developed for sequences!


%\end{multicols}

%\hrule
\vspace*{.1in}

\subsection*{Geometric Series:} 
A series of the form\\
\( a+ ar+ ar^2+ ar^3+ \ldots+ ar^{n-1}+\ldots = \sum_{n=1}^\infty a r^{n-1}\)\\
is called a \textbf{geometric series}. \\~\\
A geometric series converges to \(S=\frac{a}{1-r}\) if \(|r|<1\), and diverges otherwise.



%%\end{multicols}

%\textbf{Theorem 8.2.6:}\\~\\
%If the series \(\Sum a_n\) is convergent, then \(\lim_{n\rightarrow\infty} a_n = 0\).\\~\\~\\
%
%%\columnbreak
%
%\underline{\textbf{Test for Divergence:}}\\
%If \(\lim_{n\rightarrow\infty} a_n\) DNE \textbf{-OR-} \(\lim_{n\rightarrow\infty} a_n \neq 0\), then \(\Sum a_n\) is  divergent.\\~\\~\\
%
%
%
%\textbf{Theorem 8.2.8:}\\~\\
%If \(\sum a_n\) and \(\sum b_n\) are convergent series, then so are:
%\begin{enumerate}[(i)]
%\item \(\sum c a_n = c\sum a_n\)
%\item \(\sum (a_n+ b_n) = \sum	a_n + \sum	b_n\)
%\item \(\sum (a_n- b_n) = \sum	a_n - \sum	b_n\)
%\end{enumerate}


%
%\underline{\textbf{The Integral Test:}}\\~\\
%Suppose \(f\) is a continuous, positive, decreasing function on \([1,\infty)\) and let \(a_n = f(n)\). Then the series \(\Sum a_n\) is convergent if-and-only-if the improper integral \(\int_1^\infty f(x) dx\) is convergent:
%\begin{enumerate}[(i)]
%\item If \(\int_1^\infty f(x) dx\) is convergent, then \(\Sum a_n\) is convergent.
%\item If \(\int_1^\infty f(x) dx\) is divergent, then \(\Sum a_n\) is divergent.
%\end{enumerate}
%
%\vspace*{.2in}
%
%The \textbf{p-series} \(\Sum \frac{1}{n^p}\) is\\
%\begin{itemize}
%\item convergent if \(p>1\)
%\item divergent if \(p\leq 1\)
%\end{itemize}

%%\end{multicols}

%\end{framed}

%\end{minipage}
\section*{Proof of the Geometric Series Convergence Result:}
(This will be filled in together in class)

\pagebreak

\section*{Geometric Series Examples \& Problems}


Determine whether the geometric series is convergent or divergent. If it is convergent, find its sum.

\begin{enumerate}
%\addtocounter{enumi}{2}

%\item 
%%\begin{multicols}{2}
%\begin{enumerate}
\item \(\qquad 2+6+18+54+162+\ldots\) \vfill
\item \(\qquad2+1+\frac{1}{2}+\frac{1}{4}+\frac{1}{8}+\frac{1}{16}+\ldots\) \vfill
%\end{enumerate}
%%\end{multicols}
%\vfill

%Determine whether the geometric series is convergent or divergent. If it converges, find its sum.
%\begin{enumerate}
%\addtocounter{enumi}{4}
\item \(\qquad\frac{1}{3}+\frac{1}{27}+\frac{1}{243}+\ldots\)\vfill
\item \(\qquad\sum_{n=1}^\infty	\frac{3^n}{\pi^{n+1}}\)\vfill

\end{enumerate}
%
%\pagebreak
%
%\section*{Problems for Group Work:}
%\textbf{Be sure to fully justify your reasoning as a part of your solutions.}\\
% The answers are upside-down on the bottom of this page.
%
%\begin{enumerate}
%
%\item Determine the convergence/divergence of the following series: \label{prob3}
%\begin{enumerate}[a)]
%
%\item \(\sum_{n=1}^\infty \ \frac{1}{k^n}, \quad k>1\)\vfill
%
%\item \(\sum_{n=1}^\infty \ \frac{4\cdot 5^n - 5\cdot 4^n}{6^n}\)\vfill
%
%\item \(\sum_{n=1}^\infty \ (-1)^n\)\vfill
%
%\item \(\sum_{n=1}^\infty \ \sin\left(\frac{n}{n+1}\right)\)\vfill
%
%\item \(\sum_{n=1}^\infty \ (-1)^{2n}\)\vfill
%
%%\item \(\sum_{n=1}^\infty \ \left[\frac{5}{n(n+1)}-\left(-\frac{1}{2}\right)^n\right]\)\vfill
%
%\end{enumerate}
%Solution: a) Converge, b) Converge, c) Diverge, d) Diverge, e) Diverge, f) Converge


%\item If the improper integral \(\int_5^\infty \frac{dx}{x^p}\) converges, which of the following series \underline{must} converge?\label{prob4}
%%\begin{multicols}{3}
%\begin{enumerate}[A)]
%\item \(\sum_{n=1}^\infty \ \frac{1}{n^{p+1}}\)
%
%\item \(\sum_{n=5}^\infty \ \frac{1}{n^{p+1}}\)
%
%\item \(\sum_{n=1}^\infty \ \frac{1}{n^{p-1}}\)
%
%\item \(\sum_{n=5}^\infty \ \frac{1}{n^{p-1}}\)
%
%\item Both A and B
%
%\item Both C and D
%
%\end{enumerate}
%%\end{multicols}


%\end{enumerate}
%
%\vfill
%
%\rotatebox{180}{
%%\begin{minipage}{\textwidth}
%\underline{Answers:}\\
%%\textbf{Problem \ref{prob1}:} a) Diverge, b) Converge to 4, 
%%\textbf{Problem \ref{prob2}:} a) Converges to \(\frac{3}{8}\), b) Converges to \(\frac{3}{\pi(\pi-3)}\), 
%\textbf{Problem \ref{prob3}:} a) Converge, b) Converge, c) Diverge, d) Diverge, e) Diverge, %f) Converge, 
%%\textbf{Problem \ref{prob4}:} E,
%%\end{minipage}
%}

\end{document}
